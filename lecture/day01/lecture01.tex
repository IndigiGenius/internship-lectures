% $Header$

\documentclass{beamer}

% Day 01 Lecture 01: Introduction to FLAIR

\mode<presentation>
{
  \usetheme{Hannover}
  % or ...
  %\setbeamercovered{transparent}
  % or whatever (possibly just delete it)
}


\usepackage[english]{babel}
% or whatever

\usepackage[utf8]{inputenc}
% or whatever

\usepackage[T1]{fontenc}
% Or whatever. Note that the encoding and the font should match. If T1
% does not look nice, try deleting the line with the fontenc.


\title[Intro to FLAIR] % (optional, use only with long paper titles)
{Introduction to FLAIR}

\subtitle{Language Revitalization and Reclamation Utilizing ASR} % (optional)

\author % (optional, use only with lots of authors)
{Shawn Tsosie}
% - Use the \inst{?} command only if the authors have different
%   affiliation.

\institute[FLAIR] % (optional, but mostly needed)
{
  Applied Machine Learning,
  First Languages AI Reality}
% - Use the \inst command only if there are several affiliations.
% - Keep it simple, no one is interested in your street address.

\date % (optional)
{June 10, 2024}

\subject{FLAIR History}
% This is only inserted into the PDF information catalog. Can be left
% out. 

% If you have a file called "university-logo-filename.xxx", where xxx
% is a graphic format that can be processed by latex or pdflatex,
% resp., then you can add a logo as follows:

\pgfdeclareimage[height=0.5cm]{university-logo}{flair_dragonfly}
\logo{\pgfuseimage{university-logo}}



% Delete this, if you do not want the table of contents to pop up at
% the beginning of each subsection:
\AtBeginSubsection[]
{
  \begin{frame}<beamer>{Outline}
    \tableofcontents[currentsection,currentsubsection]
  \end{frame}
}


% If you wish to uncover everything in a step-wise fashion, uncomment
% the following command: 

\beamerdefaultoverlayspecification{<+->}


\begin{document}

\begin{frame}
  \titlepage
\end{frame}

\begin{frame}{Outline}
  \tableofcontents[pausesections]
  % You might wish to add the option [pausesections]
\end{frame}


% Since this a solution template for a generic talk, very little can
% be said about how it should be structured. However, the talk length
% of between 15min and 45min and the theme suggest that you stick to
% the following rules:  

% - Exactly two or three sections (other than the summary).
% - At *most* three subsections per section.
% - Talk about 30s to 2min per frame. So there should be between about
%   15 and 30 frames, all told.

\section{Introduction}

\begin{frame}{First Languages AI Reality}
  \begin{itemize}
    \item Who we are:
    \begin{itemize}
      \item Caroline Running Wolf
      \item Michael Running Wolf
      \item Conor Quinn
      \item Shawn Tsosie
    \end{itemize}
    \item What are we doing:
    \begin{itemize}
      \item Indigenous Language Reclamation and Revitalization
      \item Using Linguistics
      \item Using XR
      \item Using Machine Learning
    \end{itemize}
    \item Why are we doing this:
    \begin{itemize}
      \item
    \end{itmeize}
  \end{itemize}
\end{frame}

\section{Prehistory}

\subsection[Origins]{Origins}

\begin{frame}{Caroline's and Michael's Initial Work}{Initial XR Experiences.}
  % - A title should summarize the slide in an understandable fashion
  %   for anyone how does not follow everything on the slide itself.

  \begin{itemize}
  \item
    Buffalo Tongue
  \item
    Standing Rock
  \item
    Hua Ki'i
  \end{itemize}
\end{frame}

\begin{frame}{Montreal, 2019.}
  \begin{itemize}
  \item
    Worked with Kwakiutl
  \item
    Sara Child, Sanyakola Foundation
  \item    
    National Research Council (NRC) Canada in Montreal
  \item
    Goal was to see how NRC could help with ASR
  \end{itemize}
\end{frame}

\subsection{Second Subsection}

\begin{frame}{Make Titles Informative.}
\end{frame}

\begin{frame}{Make Titles Informative.}
\end{frame}



\section*{Summary}

\begin{frame}{Summary}

  % Keep the summary *very short*.
  \begin{itemize}
  \item
    The \alert{first main message} of your talk in one or two lines.
  \item
    The \alert{second main message} of your talk in one or two lines.
  \item
    Perhaps a \alert{third message}, but not more than that.
  \end{itemize}
  
  % The following outlook is optional.
  \vskip0pt plus.5fill
  \begin{itemize}
  \item
    Outlook
    \begin{itemize}
    \item
      Something you haven't solved.
    \item
      Something else you haven't solved.
    \end{itemize}
  \end{itemize}
\end{frame}


\end{document}
