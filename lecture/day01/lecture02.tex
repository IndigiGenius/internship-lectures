% $Header$

\documentclass{beamer}

% This file is a solution template for:

% - Giving a talk on some subject.
% - The talk is between 15min and 45min long.
% - Style is ornate.



% Copyright 2004 by Till Tantau <tantau@users.sourceforge.net>.
%
% In principle, this file can be redistributed and/or modified under
% the terms of the GNU Public License, version 2.
%
% However, this file is supposed to be a template to be modified
% for your own needs. For this reason, if you use this file as a
% template and not specifically distribute it as part of a another
% package/program, I grant the extra permission to freely copy and
% modify this file as you see fit and even to delete this copyright
% notice. 


\mode<presentation>
{
  \usetheme{Warsaw}
  % or ...

  \setbeamercovered{transparent}
  % or whatever (possibly just delete it)
}


\usepackage[english]{babel}
\usepackage[latin1]{inputenc}
\usepackage[T1]{fontenc}
\usepackage{listings}


\title[Internship Goals] % (optional, use only with long paper titles)
{Internship Work Goals}

\subtitle
{} % (optional)

\author % (optional, use only with lots of authors)
{Shawn Tsosie\inst{1}}
% - Use the \inst{?} command only if the authors have different
%   affiliation.

\institute[FLAIR] % (optional, but mostly needed)IR
{
  Applied Machine Learning, \\
  First Languages AI Reality}
% - Use the \inst command only if there are several affiliations.
% - Keep it simple, no one is interested in your street address.

\date % (optional)
{June 10, 2024 / 2024 FLAIR Internship}

\subject{Work Breakdown}
% This is only inserted into the PDF information catalog. Can be left
% out. 



% If you have a file called "university-logo-filename.xxx", where xxx
% is a graphic format that can be processed by latex or pdflatex,
% resp., then you can add a logo as follows:

% \pgfdeclareimage[height=0.5cm]{university-logo}{university-logo-filename}
% \logo{\pgfuseimage{university-logo}}



% Delete this, if you do not want the table of contents to pop up at
% the beginning of each subsection:
\AtBeginSubsection[]
{
  \begin{frame}<beamer>{Outline}
    \tableofcontents[currentsection,currentsubsection]
  \end{frame}
}


% If you wish to uncover everything in a step-wise fashion, uncomment
% the following command: 

%\beamerdefaultoverlayspecification{<+->}


\begin{document}

\begin{frame}
  \titlepage
\end{frame}

\begin{frame}{Outline}
  \tableofcontents
  % You might wish to add the option [pausesections]
\end{frame}


% Since this a solution template for a generic talk, very little can
% be said about how it should be structured. However, the talk length
% of between 15min and 45min and the theme suggest that you stick to
% the following rules:  

% - Exactly two or three sections (other than the summary).
% - At *most* three subsections per section.
% - Talk about 30s to 2min per frame. So there should be between about
%   15 and 30 frames, all told.

\section{Introduction}

\subsection{Software Components}

\begin{frame}{Software inputs and outputs}
  What will the software look like?
  \begin{itemize}
  \item
    training
    \begin{itemize}
    \item
    intput: \texttt{python train.py config.json}
    \item
    output: Model suitable for inference
    \end{itemize}
  \item
  inference
    \begin{itemize}
    \item
    intput: \texttt{python inference.py audio\_file.wav}
    \item
    output: \texttt{audio\_file\_transcription.xml}
    \end{itemize}
  \end{itemize}
\end{frame}

\subsection[Machine Learning Breakdown]{Major Machine Learning Components}

\begin{frame}{Data, model, optimizer, inference.}
  % - A title should summarize the slide in an understandable fashion
  %   for anyone how does not follow everything on the slide itself.

  Any machine learning project can be broken down into 4 major components
  \begin{itemize}
  \item
    Data
  \item
    Model
  \item
    Optimizer
  \item
    Inference
  \end{itemize}
\end{frame}


\section{Data, Model, Optimizer, Inference}

\begin{frame}{Overview pt. 1}

  Each component has several components.
  \begin{itemize}
  \item 
    Data
    \begin{itemize}
    \item Normalization
    \item Tokenization
    \item Data transforms
    \item Batching
    \end{itemize}
  \item 
    Model
    \begin{itemize}
    \item architecture
    \item initialization, e.g., fine-tuned or not
    \item LoRA module
    \item Quantization
    \item Acoustic Model
    \item Language Model
    \end{itemize}
  \end{itemize}

\end{frame}

\begin{frame}{Overview pt. 2}
  \begin{itemize}
  \item 
    Optimizer
    \begin{itemize}
    \item learning rate
    \item learning rate scheduler
    \item optimization algorithm
    \end{itemize}
  \item 
    Inference
    \begin{itemize}
    \item acoustic model
    \item language model
    \item decode function
    \end{itemize}
  \end{itemize}
\end{frame}

\begin{frame}{Data}

  The first component will be \texttt{data.py} file that gets the data and serves it.
  Currently, this is the \texttt{pseudo-PhoNet} library.
  Some components we'll have to work on are:
  \begin{itemize}
  \item a \texttt{collate\_fn} function that batches the data together.
  \item text cleaning
  \item normalization of text
  \item tokenization of text
  \end{itemize}
\end{frame}

\begin{frame}{Model}

  The second component will be \texttt{model.py} file that gets the model and instantiates it.
  This should be relatively easy compared to the other components as we are not planning on creating any new
  architectures ourselves.
  \begin{itemize}
  \item model architecture
  \item choosing which LoRA modules to use
  \end{itemize}
\end{frame}

\begin{frame}[fragile]
\frametitle{Optimizer}

  The third component will be \texttt{train.py} file that imports from \texttt{model.py}
  and \texttt{data.py} and trains a model on it.
  While the basic concept of this file will be simple, it will house the majority of the
  hyperparameters that we will use.
  As an example of our start, a basic \texttt{torch} training loop is:
  \begin{itemize}
  \item 
  \begin{lstlisting}[language=Python]
  for epoch in range(epochs):
      for src, tgt in dataloader:
          opt.zero_grad()
          pred = model(src)
          loss = loss_fn(pred, tgt)
          loss.backwad()
          opt.step()
  \end{lstlisting}
  \end{itemize}
\end{frame}

\begin{frame}{Inference}

  The final component will be \texttt{inference.py} file that gets the data and serves it.
  Currently, this is the \texttt{pseudo-PhoNet} library.
  Some components we'll have to work on are:
  \begin{itemize}
  \item 
  \end{itemize}
\end{frame}

\begin{frame}{Make Titles Informative.}
\end{frame}

\section*{Summary}

\begin{frame}{Summary}

  % Keep the summary *very short*.
  \begin{itemize}
  \item
    Their \alert{typically four components} of a machine learning project.
  \item
    The focus is on the \alert{data and model} portions of the project.
  \end{itemize}
  
\end{frame}


\end{document}
